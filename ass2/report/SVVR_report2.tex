\documentclass{article}
\usepackage[utf8]{inputenc}
\usepackage{graphicx,amsmath,amsfonts,amssymb}

\begin{document}
\title{SVVR Assignment 2}
\author{Sander Nugteren (6042023) \and Merel de Groot (6103677)}
\renewcommand{\today}{November 17, 2014}
\maketitle

\section{Visualization Pipeline}
The assignment is to create a visualization pipeline with VTK. The source will be a dataset consisting of images from a CT scanner. These 2D images combined form a 3D image of a body. Using VTK we will create a pipeline that takes the dataset as input and creates a 3D visualization of the body. The order of the sections will correspond to the steps of the pipeline.

\subsection{Source}
The data consists of 94 images created by a CT scanner. Each file is 131,072 bytes and the images are 256x256 or 512x512 pixels in size. Since 131,072 devided by 512x512 is only a half, we know that this is not possible. Deviding 131,072 by 256x256 however, gives us 2 bytes, which is a reasonable size of data per pixel. The collection of binary images is read by using \textsc{VTKImageReader2}.

\subsection{Filter: Contour}
Using the iso-surface contour visualization. 
\\ - value range for the contour value: 0 to $2^{16}$
\\ - render an isosurface contour of the dataset, add an interactor so that you can interact with the resulting visualization
\\ - allow the initial contour value to be passed from the commandline so that you can pass a different value each time you run your program > effect?

\subsection{Mapper}

\subsection{Actor: Colours}
Default: object in a dark blue on a black background
\\ Three ways to change the colour of the object:
\begin{itemize}
\item Tell the contour filter stage to not compute scalar values and then set a colour in the actor
\item Tell the mapper stage to ignore scalar values and then set a colour in the actor
\item Tell the mapper what the actual scalar range is. 
\end{itemize}

Next, change the background colour and increase the size of the render window as well.

\subsection{Data Spacing}
Configure “data spacing” in source. By default voxels are isotropic, but we need ??

\subsection{Renderer: Setting the Scene}
The renderer will create a camera and configure it so that the visualization will be visible in the render window. However, in most cases this will be suboptimal; the
initial camera angle will not provide a useful view on your visualization. In that case it will be necessary to manipulate the camera. VTK uses two models to manipulate the camera:
\\ - Camera is focused at a focal point (your visualization) and moves around
this focal point using the Elevation, Roll and Azimuth methods.
\\ - The movement of the camera is centered at the position of the camera and the orientation of the camera is controlled using the Yaw, Roll and Pitch methods.
\\ Determine a better camera angle and add an outline to your visualization that shows a bounding box to represent the extent of the input data.

\section{Results}
Explore all contour values in the dataset. 

\end{document}
